\documentclass[letterpaper,11pt]{refart}
\usepackage{hyperref}
\usepackage{amssymb,amsmath,mathtools}
\usepackage{enumitem}
\usepackage{framed}
\usepackage[normalem]{ulem}

\hypersetup{colorlinks,urlcolor=blue,linkcolor=black,citecolor=black}

\newcommand*{\email}[1]{\href{mailto:#1}{\nolinkurl{#1}}} 
\newcommand{\srourl}{\url{https://stellarcollapse.org/SROEOS}}
\newcommand{\LS}{L\&S }
\newcommand{\LSn}{L\&S}
\newcommand{\SRO}{SRO }
\newcommand{\SROn}{SRO}
\newcommand{\todo}[1]{{\color{BurntOrange}\bf $\blacksquare$~{[TODO: #1]}}}
\newcommand{\lime}[1]{{\textcolor{LimeGreen}{#1}}}
\newcommand{\sch}[1]{{\color{OrangeRed}$\blacksquare$~\textsf{[A COMMENT: #1]}}}

\usepackage[dvipsnames]{xcolor}
\usepackage{newverbs}

\usepackage{tocloft}
\usepackage{etoolbox}
\patchcmd{\thebibliography}{\section*}{\section}{}{}

\makeatletter
\renewcommand\subsubsection{\@startsection{subsubsection}{3}{\z@}%
                           {-3.25ex\@plus -1ex \@minus -.2ex}%
                           {0.3ex \@plus .2ex}%
                           {\normalfont\normalsize\bf\fontsize{11}{13}\selectfont}}
\makeatother

\newverbcommand{\verbfile}{\color{red}}{}
\newverbcommand{\verbexec}{\color{magenta}}{}
\newverbcommand{\verbnml}{\color{cyan}}{}
\newverbcommand{\verbprm}{\color{ForestGreen}}{}

\newenvironment{namelist}
  {
    \VerbatimEnvironment
    \vskip\baselineskip\hrule
    \begin{Verbatim}[xleftmargin=0pt]%
  }
  {\end{Verbatim}\hrule\vskip\baselineskip}

\renewcommand\abstractname{Abstract}
\oddsidemargin = 1.5in
\settextfraction {1.0}


\title{The Schneider-Roberts-Ott Equation of State Code}
\author{Andre da Silva Schneider (\email{andschn@caltech.edu}) 
     \\ Luke F. Roberts (\email{robertsl@nscl.msu.edu})
     \\ Christian D. Ott (\email{cott@tapir.caltech.edu})
     \\ \url{https://bitbucket.org/andschn/sroeos}
     \\ \url{https://stellarcollapse.org/SROEOS}
     \\ \email{SROEOS@stellarcollapse.org}
       }

\date{}

\begin{document}
\maketitle


\noindent \begin{flushright}{Last updated \date{\today}.}\end{flushright}

\begin{abstract}

This is the user guide to the Schneider-Roberts-Ott (SRO) equation of
state code described in Schneider, Roberts, \& Ott (2017)
\cite{schneider:17}.  Here we provide details on how to compile and
run the various parts of the SRO code. We also discuss the details of
their inputs and outputs.
\end{abstract}

\tableofcontents


\section{Introduction}

The Schneider-Roberts-Ott (SRO) equation of state (EOS) code is
intended to build EOS tables of hot dense matter for astrophysical
simulations. The development of the SRO EOS was supported by the
National Science Foundation Theoretical and Computational Astrophysics
Network (TCAN) award NSF AST-1333520.


Our work, described in Reference \cite{schneider:17} (SRO hereafter),
builds upon the seminal work of Lattimer and Swesty (\LS hereafter)
\cite{lattimer:91} and presents a generalized method to generate EOSs
using a compressible non-relativistic liquid-drop model with Skyrme
interactions.


At low densities, one may choose to transition from the \LSn-type EOS,
which uses a single nucleus approximation (SNA), to a nuclear
statistical equilibrium (NSE) description of an ensemble of nuclei. To
simplify its usage, the SRO code is separated into three different
parts:
\begin{enumerate}[label={(\arabic*)}]
 \item The single nucleus approximation (\texttt{SNA}) code that builds a
   nuclear EOS using a compressible non-relativistic liquid-drop model
   with Skyrme interactions. The \texttt{SNA} code does not add electrons,
   positrons, and photons.
 \item The nuclear statistical equilibrium (\texttt{NSE}) code, which
   builds a nuclear EOS for a mixture of non-interacting nuclei.  The
   \texttt{SNA} code also does not add electrons, positrons, and photons.

 \item The \texttt{MERGE} code which combines the SNA and/or the NSE
   EOSs and adds electrons, positrons, and photons. If so desired,
   \texttt{MERGE} can also run on pure SNA or pure NSE tables.

   \textbf{IMPORTANT:} We recommend that the user read the description
   of the \texttt{MERGE} code first before generating SNA and NSE
   tables.  Care must be taken when setting up the SNA and NSE table
   resolutions and ranges. This is discussed in
   Section~\ref{sssec:MERGE_phase}.
   
\end{enumerate}
The reason for our choice of having three separate parts is that one
may want to separately build SNA and NSE tables and change the
parameters used to merge different EOSs without having to rebuild
them.  This is discussed in detail in Section VII.A.\ of
SRO~\cite{schneider:17}.


The SRO code is written using a modern, modular, and OpenMP-threaded Fortran
90+ framework and is made publicly available at \srourl\, under the
\textsc{Gnu General Public License} version 3 (GPL 3).

Some files and subroutines used in the SRO code are from other
open-source codes and were slightly modified to suit our needs. These
are:
\begin{enumerate}
 \item The Timmes \& Arnett EOS is used to compute the EOS of electrons and photons. For details see Appendix A of SRO and Reference \cite{timmes:99}. 
 \item The Fermi integrals and their inverses are computed using the subroutines of Fukushima \cite{fukushima:15,fukushima:15b}.
 \item To solve the system of Equations (38) in SRO, we use the equation solver package ``\textit{nleqslv}'' of Hasselman \cite{hasselman:16}. 
 \item Derivatives of thermodynamical quantities w.r.t. parameters that are a function of coupled independent variables are computed using the LU decomposition subroutines of the \LS code available at \url{http://www.astro.sunysb.edu/dswesty/lseos.html}. 
\end{enumerate}



Besides the \SRO code, we also provide open-source Fortran 90 and C++
HDF5 table readers that interpolate between calculated values.

\section{Disclaimer and Getting Help}

\begin{framed}
The SRO EOS code is open source under the GNU General Public License
version 3 and may be used at your own risk. The code comes with
absolutely no warranty. While we will always try to help users, we are
unable to guarantee that we will be able to provide support or help if
you run into problems running it. If you decide to use the SRO EOS
code or tables generated with it in published work, it is your
responsibility to test the code and ensure its physical correctness
and consistency in the application problem you are using it for.
\end{framed}

As a prerequisite to running the SRO EOS code, you should be familiar
and comfortable with the Linux/Unix-style command line and file
systems (including symbolic links), with compilers and Makefiles, and
with installing external libraries either by hand or via your
operating system's package manager.

If you run into physics problems, find bugs, or inconsistencies, please
let us know by sending email to this address:

\begin{center}
\texttt{SROEOS@stellarcollapse.org} 
\end{center}

Please understand that our resources and time are limited and we may
not be able to help you with basic compilation problems, coding
problems, or with problems with external libraries.

\section{Requirements, Dependencies, and \texttt{make.inc} for Compilation}
\label{sec:req}

When making EOS tables, we recommend that you use a computer system
with ample compute power (many CPU cores) and with at least 32 GB of
RAM to build EOS tables. The default table ranges and resolutions in the
SNA and NSE parts lead to tables of around 7 GB that will be merged
to a final standard-resolution table of around 700 MB. The SNA code
is the slowest component and typically requires a few hours using 20 CPU
cores to build the default SNA table.



The SRO EOS code requires the following software on your system:

\begin{enumerate}
\item The git version control system. We recommend git version 2 and
  higher. If it is not already installed, try installing it via
  your operating system's package manager or get it from
  \url{https://git-scm.com}.
 \item Fortran 90+, C++, and C compilers, such as the \verb|gcc|
   compiler suite available at \url{https://gcc.gnu.org} or through
   your operating system's package manager. It is highly recommended
   to use a compiler that supportes OpenMP shared-memory OpenMP
   parallelization (most current ones do). Without parallelization,
   solving for the EOS will be very slow. Make sure to look up the
   correct compiler flag to enable OpenMP support during compilation.

   Important note: The SRO EOS uses some Fortran 2003 features such as
   IEEE Arithmetic and ISO C Bindings. Because of this, recent compiler versions
   are necessary. For \verb|gcc|, this is any version
   $\ge 5$.
   
 \item The \verb|HDF5| library, available at
   \url{https://support.hdfgroup.org/HDF5/}. It must be compiled with
   Fortran 90+ support.

 \item The \verb|BLAS| library, available at
   \url{http://www.netlib.org/blas/} or through your operating
   system's package manager.

 \item The \verb|LAPACK| library, available at
   \url{http://www.netlib.org/lapack/} or through your operating
   system's package manager.

\end{enumerate}
Note that, in general, the \verb|HDF5|, \verb|LAPACK|, and \verb|BLAS|
libraries must be compiled with the same compiler(s) (and compiler
version(s)) used to compile the SRO EOS code components.

Should you decide to compile some of the above libraries yourself, we
strongly recommend that you disable shared libraries and go with
static libraries instead. This will greatly reduce potential headaches
with dynamically linked symbols that are not found. Also, as a general
guideline, if you can link the SRO EOS code components statically, do
so. Linking statically may not be possible on MacOS and may require
the installation of ``development'' packages for compilers and libraries
on most Linux systems.

Once you have setup all software/libraries, you need to create the
\texttt{make.inc} file in the root directory of the SRO EOS code. This
file is read by the SRO EOS Makefiles to compile the code components
\texttt{SNA}, \texttt{NSE}, and \texttt{MERGE}. We provide various
examples for Linux and MacOS systems in the directory \textcolor{red}{\texttt{make\_inc\_examples}}.


\section{The Single Nucleus Approximation Code \texttt{SNA}}\label{sec:SNA}

The \texttt{SNA} code (residing in the \texttt{SNA} subdirectory) is
used to compute the EOS of hot dense matter using a compressible
non-relativistic liquid-drop model with Skyrme interactions. It does
not include electrons, positrons, and photons.

The \texttt{SNA} code computes the most favorable composition of
nuclear matter for a given Skyrme parametrization for the
nucleon-nucleon interactions for a range of proton fraction $y$
(identical to electron fraction $Y_e$ because of charge neutrality),
temperature $T$, and number density $n$. A discussion of the code
library dependencies, its input and output as well as units used are
detailed in the rest of this section.


\subsection{Compiling and Running the \texttt{SNA} Code}\label{ssec:SNA_dep}

Make sure your system satisfies the requirements outline in
\S\ref{sec:req} and that your \texttt{make.inc} file is set up
correctly.

Compiling the \verb|SNA| code should now be as simple as typing on
your terminal from the root directory of the SRO EOS:

\begin{verbatim}
 > cd SNA
 > make
\end{verbatim}


This will create two executables in the \verbfile|SNA| directory,
\verbexec|main| and \verbexec|test|.  Their inputs and outputs are
discussed in Sections \ref{ssec:SNA_in} and \ref{ssec:SNA_out},
respectively.

\textbf{IMPORTANT:} When solving for the EOS, the code will make use
of OpenMP parallelization (provided the compiler supports it and the
correct flags are set during compilation). It will use as many OpenMP
threads as specified by the system environment variable
\texttt{OMP\_NUM\_THREADS}. Make sure to check what this is set to and
adjust it, if you want to use less or more threads. We do not
recommend using more threads than available physical CPU cores on your
machine.



We carried out tests of both executables using both the \texttt{GNU}
and the Intel compiler suite.  Even though all computations are
performed using double precision arithmetic, small fluctuations in
some of the results, usually of $\mathcal{O}(10^{-7})$ or less, may
occur depending on the compiler used and the optimization level flag
\verb|-O#| chosen in the \texttt{make.inc}. Experiments show that
these small differences result from the computation of some second
derivatives when obtaining surface properties of nuclear matter.


\subsection{Input}\label{ssec:SNA_in}

The \texttt{\textcolor{magenta}{main}} and
\texttt{\textcolor{magenta}{test}} executables take two input files,
with one of the input files being the same for both:
\verbfile|input/skyrme.in|.  The \verbexec|main| executable uses as its 
second input the \verbfile|input/space.in|
file, which contains details on the ranges and resolutions in density,
temperature, and proton fraction for the resulting EOS. The
\verbexec|test| executable uses as its second input the
\verbfile|input/test.in| file, which contains a single point in
density, temperature, and proton fraction, and, if desired, an initial
guess for the solution.  We discuss these input files in the
following.




\subsubsection{The Skyrme input \textcolor{red}{\texttt{input/skyrme.in}}}
\label{sssec:skyrme.in}

\sloppy The \verbfile|input/skyrme.in| file contains four required and
three optional Fortran namelists that specify the Skyrme
parametrization being used and other EOS parameters and code control
flags.  The code will read \verbfile|input/skyrme.in|, but the file
itself is a symbolic link.  By default it points to
\verbfile|input/list_SKRA.in|, which is the file specifying the SKRA
parametrization.  If a different parametrization is desired, the
symbolic link can be changed to point to one of the other
parameterizations that we provide (or to a user-customized file).  In
the \verbfile|input| directory, we provide parameter sets for the
following parametrizations (as described in the SRO EOS paper
\cite{schneider:17}): KDE0v1, LNS, LS220, LS220*, NRAPR, SKRA, SkT1,
SkT1$^*$, Skxs20, SLy4, and SQMC700. See the SRO paper
\cite{schneider:17} for a discussion of these parametrizations.

The Fortran namelists in the \verbfile|input/skyrme.in| file are
\begin{enumerate}
 \item \verbnml|TABLE_INPUT| (required),
 \item \verbnml|SKYRME_TYPE| (required),
 \item \verbnml|SKYRME_COEFFICIENTS| (required),
 \item \verbnml|SURFACE_PROPERTIES| (required),
 \item \verbnml|HEAVY_NUCLEI_SIZE| (optional), 
 \item \verbnml|STIFFEN_EOS| (optional),
 \item \verbnml|TABLE_OUTPUT| (optional).
\end{enumerate}

\bigskip

\textbf{The \texttt{\color{cyan} TABLE\_INPUT} Namelist}


The \verbnml|TABLE_INPUT| namelist contains information about the type
of output files to write, their names, and locations.  An example of
the \verbnml|TABLE_INPUT| namelist, taken from
\verbfile|input/list_SKRA.in| file, is the following:

{\color{cyan}
\begin{verbatim}
&TABLE_INPUT
  filename_base     = "SKRA",
  output_directory  = "EOS_SKRA",
  make_hdf5_table   = .true.,
  write_solutions_to_file = .false.,
  redefine_print_parameters = .false.,
/
\end{verbatim}}
Note that the only parameter of this list relevant to the
\verbexec|test| executable is \verbprm+output_directory+ as that is
where the code's output, if any, is written.  For the \verbexec+main+
executable, the only required parameters are
\verbprm+output_directory+ and
\verbprm+filename_base+. The logical parameters are set by default to
the values given above and only need to be explicit if changed. The
parameter \verbprm+redefine_print_parameters+ is used to limit the
quantities printed to the SNA EOS table. Its usage is described
alongside the \verbnml+TABLE_OUTPUT+ namelist below.


The parameter \verbprm|make_hdf5_table| sets whether to write an 
\verb|HDF5| table in the \verbfile|output_directory| directory.  
The table name starts with \verbprm|filename_base| followed by 
information about the table resolution in $\log_{10}$(density), 
$\log_{10}$(temperature), and proton fraction. Next is information on 
the git revision. If the code to make the table was compiled from a 
repository with local changes, the short git hash is prepended with a 
captial ``M''. If the repository is all fully committed, the short git 
hash has no prefix. Finally, the name contains the date the table was
generated. For example, using the default
\verbfile|input/skyrme.in| and \verbfile|input/space.in| files,
a table name could look lik this:
\verbfile|SKRA_rho427_temp169_ye67_gitMd089cf6_20170624.h5|.  Note
that the table is only written after all solutions have been
computed. Also note that we are including the full source code and all
inputs needed to generate the table. See Section~\ref{sec:formaline}
for details on this. 


The details of the contents (including units) of the \verb|HDF5| output file 
is discussed in Section~\ref{sssec:SNA_main_out}.

  
The
\verbprm|write_solutions_to_file| variable determines whether to write the independent
variables that solve the system of equations and other debugging
information for each point in phase space to an ASCII file. By the
default it is set to
\verbprm|false|. If \verbprm|write_solutions_to_file = .true.|, the
code creates the directories \verbfile|SOL| and \verbfile|NO_SOL|
inside the \verbfile|output_directory|.  An output file for each
proton fraction $y$ is created in
\verbfile|SOL| in which the solutions to Equations (38) in SRO \cite{schneider:17}, as well as the solution for
the uniform system discussed in Section \ref{ssec:SNA_out}, are
written for each temperature and density.  To debug issues at density,
temperature, and proton fraction points where no solutions were found,
an output file for each proton fraction $y$ is created in
\verbfile|NO_SOL|.  If no solution is found for a given point, a
detailed error of the likely issue is stored in the corresponding $y$
file.  We provide a detailed description of the output files in
Section \ref{ssec:SNA_out}.


\bigskip
\textbf{The \texttt{\color{cyan} SKYRME\_TYPE} Namelist}

The \verbnml|SKYRME_TYPE| namelist is used to give a few details on
the input Skyrme parametrization. An example is shown below.

{\color{cyan}
\begin{verbatim}
&SKYRME_TYPE
  Skyrme_parametrization_type  = 0,
  non_local_terms              = 1, ! (default is 1) 
  Use_default_constants        = 0, ! (default is 0) 
  High_density_stiffening      = .FALSE., ! (default is .FALSE.) 
/
\end{verbatim}}



%This namelist contains 4 inputs:
%\verbprm|Skyrme_parametrization_type|,
%\verbprm|non_local_terms|, \verbprm|Use_default_constants|, and
%\verbprm|High_density_stiffening|.

The integer parameter \verbprm|Skyrme_parametrization_type| can assume
the values 0 and 1:

\begin{itemize}

\item \verbprm|Skyrme_parametrization_type = 0| $\Rightarrow$ uses
  ``standard'' Skyrme terms, \textit{i.\,e.}, the Skyrme terms usually
  found in the literature. The standard Skyrme parameters are $x_j$,
  $t_j$ (with $j=0,1,2,3i$), and $\sigma_i$. Here, $i$ is the number
  of non-local terms given by the parameter
  \verbprm|non_local_terms|. For almost every parameterization found
  in the literature $i=1$. In the code, the ``standard'' terms are
  converted to the ``simplified'' Skyrme terms using Equation (14) of
  SRO \cite{schneider:17}.

\item \verbprm|Skyrme_parametrization_type = 1| $\Rightarrow$ reads as
  input the ``simplified'' Skyrme parameters $a$, $b$, $c_i$, $d_i$,
  $\delta$, $\alpha_1$, $\alpha_2$, and $q_{tt'}$ for $t,t'=p,n$ (see
  Equation 14 of \cite{schneider:17}). Any values not given are
  assumed to be zero. Again, $i$ is the number of non-local terms
  given by the parameter \verbprm|non_local_terms|.

\end{itemize}


The next control in the \verbnml|SKYRME_TYPE| namelist is
\verbprm|Use_default_constants|.  Its value can be 0 or 1.  If \verbprm|Use_default_constants = 0|
then experimental values for the neutron and proton masses are
used.  Also, the neutron proton mass difference $\Delta=m_n-m_p$ is
obtained self-consistently and the binding energy of alpha particles
is determined with respect to that of a gas of free neutrons,
$B_\alpha=30.8866$\,MeV.  Otherwise, if
\verbprm|Use_default_constants = 1| then $m_n=m_p$,
$\Delta=1.29$\,MeV, and $B_\alpha=28.3$\,MeV.  The latter choice is
consistent with that used by \LS \cite{lattimer:91}.


Finally,
\verbprm|High_density_stiffening| can assume the values \verbprm|.true.|
or \verbprm|.false.|.  If set to \verbprm|.false.|, no modifications to
the Skyrme parameterization given in the namelist
\verbnml|SKYRME_COEFFICIENTS| is performed.  If set to
\verbprm|.true.|, then extra terms are added to the Skyrme parametrizations
that stiffen the EOS at high densities.  The stiffening follows the
discussion of Section V A of SRO \cite{schneider:17} and is controlled
by the namelist \verbnml|SKYRME_STIFFENING| discussed below.



\textbf{The \texttt{\color{cyan} SKYRME\_COEFFICIENTS} Namelist}

The \verbnml|SKYRME_COEFFICIENTS| namelist contains the Skyrme
parameters.  Its format for the case
\verbprm|Skyrme_parametrization_type = 0| and
\verbprm|non_local_terms = 1| for the SKRA parametrization is shown
below.

{\color{cyan}
\begin{verbatim}
 &SKYRME_COEFFICIENTS
  Coeff_t0            = -2895.4000d0, ! in MeV fm^3
  Coeff_t1            =   405.5000d0, ! in MeV fm^5
  Coeff_t2            =   -89.1000d0, ! in MeV fm^5
  Coeff_t3            = 16660.0000d0, ! in MeV fm^(3+3*sigma)
  Coeff_x0            =     0.0800d0, ! dimensionless
  Coeff_x1            =     0.0000d0, ! dimensionless
  Coeff_x2            =     0.2000d0, ! dimensionless
  Coeff_x3            =     0.0000d0, ! dimensionless
  Coeff_sigma         =     0.1422d0, ! dimensionless
/
\end{verbatim}}

If the standard Skyrme parameters are given as shown here
(\verbprm|Skyrme_parametrization_type = 0|), the simplified parameters
($a$, $b$, $c_i$, $d_i$, $\delta_i$, $\alpha_1$, $\alpha_2$, and
$q_{tt'}$) are calculated using Equations (14) and (27) of SRO
\cite{schneider:17}.


In the \verbprm|Skyrme_parametrization_type = 1| case, the simplified
Skyrme parameters must be given in the \verbnml|SKYRME_COEFFICIENTS|
namelist. Parameters that are not explicitly set in the
\verbnml|SKYRME_COEFFICIENTS| namelist are assumed to be zero. An
example for the LS220 parametrization is shown in
\verbfile|input/list_LS220.in| file.




\bigskip
\textbf{The \texttt{\color{cyan} SURFACE\_PROPERTIES} Namelist}

The \verbnml|SURFACE_PROPERTIES| namelist controls how the surface
properties are obtained (see Section II.B.\ of SRO
\cite{schneider:17}). In can also contain information on the
proton-fraction dependence of the critical temperature (see Equation
22 of SRO \cite{schneider:17}).  For the SKA case, we have:


{\color{cyan}
\begin{verbatim}
 &SURFACE_PROPERTIES
  fit_surface    = .false.,
  Surface_sigma  =   1.124937302481D+00, ! in Mev fm^(-2)
  Surface_q      =   2.398029067503D+01, ! dimensionless
  Surface_p      =   1.707004256669D+00, ! dimensionless
  Surface_lambda =   3.702622653011D+00, ! dimensionless
/
\end{verbatim}}

If \verbprm|fit_surface = .false.|, then the code will use the specified
precomputed surface parameters (see Section II.B.\ of SRO
\cite{schneider:17}) as in the above. We provide precalculated surface
parameter values for all included parameterizations. Note that the
provided values may change slightly $\mathcal{O}(10^{-6} - 10^{-7})$,
depending on the compiler and optimization flag used, since the calculation
in sensitive to round-off error.  We do not
expect differences of this order to significantly alter the EOS.


If \verbprm|fit_surface = .true.|, then the code will compute the
surface parameters, which can take $\sim2 - 5$ minutes, which is
significant if one is simply interested in running the \verbexec|test|
code. \verbprm|fit_surface=.true.| must be used for any parameterization
for which the surface parameters are not known. 
Once the surface fit for a given parametrization is found, 
the user should look up its values in the 
\verbfile+output_directory/Surface_properties.dat+ file, change 
\verbprm|fit_surface = .true.|  to  \verbprm|fit_surface = .false.|
in the  \verbfile+list.in+ file and copy the values of the fit 
parameters to the  \verbnml|SURFACE_PROPERTIES|  namelist described below. 
This prevents the surface properties from being calculated every time 
the user wants to run a test or remake a table reusing a given parametrization.
For the parametrizations available in the  \verbfile+input+ directory, the 
surface fittings are already provided.



Note that in the \verbnml|SURFACE_PROPERTIES| namelist shown above,
the critical temperature $T_c(y_i)$ (Equation 22 of SRO
\cite{schneider:17}) fitting is performed by the code since, by
default, \verbprm|fit_temperature = .true.|. However, this calculation
is extremely fast we do not bother do add the precalculated values for
the parameterization in Equation (22).  However, in the
\verbfile|input/list_LS220star.in| input of the LS220$^*$ case (see
\cite{schneider:17}), we set \verbprm|fit_temperature = .false.| and
specify the critical temperature fit coefficients explicitly to use
the same fit as L\&S. 


\bigskip
\textbf{The \emph{optional} \texttt{\color{cyan} HEAVY\_NUCLEI\_SIZE}
  Namelist}



In Section 2.9 of L\&S~\cite{lattimer:91}, the size of heavy nuclei in
the translational part of the free energy is fixed to $A=A_0=60$.
Though we compute $A$ self-consistently, we also allow for an optional
a \verbnml|HEAVY_NUCLEI_SIZE| namelist in case one wants to compute
the EOS using the prescription of \LSn. This requires adding
the \verbnml|HEAVY_NUCLEI_SIZE| namelist to 
\verbfile|input/skyrme.in|:
{\color{cyan}
\begin{verbatim}
&HEAVY_NUCLEI_SIZE
  fix_heavy_nuclei_size = .true.,
  A00 = 60.d0,
/
\end{verbatim}}
Note that \verb+A00+ is the desired fixed value for heavy nuclei,
$A_0$.  If \verb+A00+ is too small (\verb+A00+$\lesssim30$) or too
large (\verb+A00+$\gtrsim120$), problems can occur in the EOS
calculation.


\newpage
\bigskip
\textbf{The \emph{optional} \texttt{\color{cyan} SKYRME\_STIFFENING}
  Namelist}

As discussed in Section V.A.\ of SRO~\cite{schneider:17}, most Skyrme
parametrizations are unable to produce neutron stars with masses as
high as those observed in nature.  Thus, we add the option to include
extra terms to the Skyrme parametrization that stiffens the EOS at
high densities. These terms are added in the namelist
\verbnml|SKYRME_STIFFENING|.  The extra terms are determined following
Section V.A.\ of SRO \cite{schneider:17} (see Equation 56).  The
stiffening exponent $\delta_2$ is given by the parameter
\verbprm+stiffening_exponent+, while the ratio shown in Equation (56)
of SRO \cite{schneider:17} is given by
\verbprm+stiffening_ratio+. Finally, the weight $w$ between the extra $c_2$ and $d_2$ terms is controlled by the parameter
\verbprm+stiffening_symmetry+ in such a way that $c_2+d_2=wc_2+(1-w)d_2$.  For the SkT1* parametrization
discussed in SRO~\cite{schneider:17} the namelist has the form {\color{cyan}
\begin{verbatim}
&SKYRME_STIFFENING
  stiffening_exponent = 5.00d0,
  stiffening_ratio    = 0.01d0,
  stiffening_symmetry = 1.00d0,
/
\end{verbatim}}



\vspace*{1cm}

\textbf{The \emph{optional} \texttt{\color{cyan} TABLE\_OUTPUT}
  Namelist}


The final (optional) namlist is \verbnml|TABLE_OUTPUT|.  
By default all possible output variables are written
to the \verb|HDF5| table.  However, by adding the \verbnml|TABLE_OUTPUT| 
namelist and selecting \verbprm|redefine_print_parameters = .true.|,
one may select a set of variables to \emph{not} be written to the
tables.  The output variables are discussed in
Section~\ref{sssec:SNA_HDF5} of this user guide.

For example, one of the written variables is the pressure $P$, which
appears in the HDF5 table as \verbprm+p+.  To not write the pressure
variable to the final table add the namelist {\color{cyan}
\begin{verbatim}
&TABLE_OUTPUT
  print_p = .false.,
/
\end{verbatim}}
For other variables simply add \verbprm|print_(variable name) = .false.|, where
\verbprm|variable name| is one of the variables discussed in 
Section~\ref{sssec:SNA_HDF5}.


\bigskip

\subsubsection{SNA test input}\label{sssec:SNA_test_in}


The \verbexec+test+ executable takes as input the namelist
\verbnml+input_test_list+ in
\verbfile+input/test.in+.  This namelist determines the point in density, 
temperature, and proton fraction
where to calculate the EOS. If desired, an initial guess can be given
for the variables that are used in the solution of the EOS for 
uniform and non-uniform nuclear matter.


{\color{cyan}
\begin{verbatim}
&input_test_list
! point in phase space
proton_fraction =  30.0D-02, ! proton fraction for test
temperature     =  1.00D-00, ! temperature in MeV for test
density         =  1.00D-03, ! density in fm^-3 for test
! use initial guesses?
guess                 = .true.,
uniform_sol_guess     = -6.5d0,
non_uniform_sol_guess = -9.0D0, -110.0D0, -2.15D0,
/
\end{verbatim}}
As discussed in \LSn~\cite{lattimer:91} the solution for the uniform system 
is solved in terms of the independent variable $y_p=n_{po}/n$ such that
\begin{equation}
n_{no}=n\frac{y_p(2-nv_\alpha(1-y))+2(1-2y)}{2-nv_\alpha y}\,.
\end{equation}

The notation used here is the same as in SRO~\cite{schneider:17}.
From the value of $y_p$ one obtains the neutron and proton densities,
$n_{no}$ and $n_{po}$, respectively.  Round-off errors make it
difficult to find solutions for uniform proton rich matter ($y>0.50$)
for most densities $n$ and temperatures $T$ when using the above
independent parameter.  Thus, unlike \LSn, for proton rich matter, we
choose the independent variable to be $y_n=n_{no}/n$, which implies
\begin{equation}
n_{po}=n\frac{y_n(2-nv_\alpha y)-2(1-2y)}{2-nv_\alpha (1-y)}\,.
\end{equation}
Thus, if desired, when running the \verbexec|test| code one may add an
initial guess to $y_p$ or $y_n$ through the parameter
\verbprm+uniform_sol_guess+. 


Similarly, a guess of the independent variables
$\zeta=[\log_{10}(n_{no}), \log_{10}(n_{po}), \log_{10}(u)]$ that
determine the solution to the non-uniform system (see Section III of
SRO \cite{schneider:17} may be added with the parameter
\verbprm+uniform_sol_guess+.

We have tried our best to implement a method that will find a solution
over the entire reasonable space in density, temperature, and proton
fraction. However, convergence cannot be guaranteed and the code may
fail for some choices of density, temperature, and proton fraction.

Also, in rare cases we have observed that the code returns a solution
which does solve the systems of equations, but returns a non-physical
condition.  We have tried to eliminate those from the output and look
for realistic solutions.  However, this is not always possible and the
user sohuld be aware that unphysical solutions, though very unlikely,
may pop up from time to time. This is we advise the user to carefully
test the resulting tables.

Finally, to avoid underflows/overflows we limit the absolute values of
some quantities to a maximum of $10^{100}$ and a minimum of
$10^{-100}$. To the best of our knowledge, this does not affect the
final solutions. Another option would be to change from double
precision to quadruple precision arithmetic. However, this would be
cumbersome and would render the code very slow.

\vspace*{-0.1cm}
Even if no initial guess is given, the SNA \verbexec+test+ code should
be able to find solutions for most parametrizations within the limits:
\begin{itemize}
 \item $10^{-14}\,\mathrm{fm}^{-3}\lesssim{n}\lesssim 10 \, \mathrm{fm}^{-3}$;
 \item $10^{-4}\,\mathrm{MeV}\lesssim{T}\lesssim 300\,\mathrm{MeV}$;
 \item $0.001\lesssim y \lesssim 0.7$.
\end{itemize}
The code can become slow at very low temperatures and
very low densities.


% Note that the uniform_sol_guess of dimension 1 and the non_uniform_sol_guess
% guesses of dimension 3 are initial guesses for the independent
% variables that solve the uniform and non-uniform system of equations,
% respectively. The solution to the uniform system depends only on a
% single independent variable, while the non-uniform system depends on
% three. The user should refer to the published paper and/or the user
% guide for details. The guesses are only used if the parameter 'guess'
% is set to '.true.'. 



\vspace*{-0.3cm}
\subsubsection{SNA Table Ranges and Resolution Input File \textcolor{red}{\texttt{input/space.in}}}\label{sssec:SNA_phase_space_in}


The \verbexec+main+ executable also takes as input the namelist
\verbnml+space_list+ contained in
\verbfile+input/space.in+.  This namelist determines the range and spacing in $\log_{10}n$ (density $n$ given in fm$^{-3}$),
$\log_{10}T$ (temperature $T$ given in MeV), and proton fraction $y$
that will be covered by the output table.  The example below should be
self-explanatory.  Note that there are two methods for determining the
spacing for each of the three variables and that we commented out the
superseding option.

{\color{cyan}
\begin{verbatim}
&space_list
yp_min      =  0.00499D0,       ! minimum proton fraction
yp_max      =  0.66499D0,       ! maximum proton fraction
steps_in_Yp =  67,              ! steps in Yp every 0.01
! Yp_spacing = 0.01D0,          ! spacing in Yp 
! (if used Yp_spacing supersedes option for steps_per_hundredth_in_yp)

log10n_min  = -13.20d0,         ! log10 of minimum density in fm^-3
log10n_max  =   1.0d0,          ! log10 of maximum density in fm^-3
steps_per_decade_in_n  =  30,   ! steps per decade in density
! log10n_spacing = 0.03333333d0,! spacing in log10(n) 
! (if used log10n_spacing supersedes option for steps_per_decade_in_n)

log10T_min  =  -3.1d0,          ! log10 of minimum density in fm^-3
log10T_max  =   2.5d0,          ! log10 of maximum density in fm^-3
steps_per_decade_in_T  = 30,    ! steps per decade in density
! log10T_spacing = 0.03333333d0,! spacing in log10(T) 
! (supersedes log10T_spacing option for steps_per_decade_in_n)
/
\end{verbatim}}



\subsection{Output}\label{ssec:SNA_out}


Both the \verbexec+test+ and the \verbexec+main+ executables produce
output with information on the employed Skyrme parametrization. Below
is a list of output files written to the directory defined by the
\verbfile+output_directory+ in the \verbnml|TABLE_INPUT| namelist.


\begin{itemize}
 \item \verbfile+Skyrme_coefficients.dat+ contains the Skyrme
   parametrization used.

 \item \verbfile+Nuclear_Properties.dat+ contains the parameters of 
   the expansion of the specific nuclear energy around nuclear saturation 
   density, Equations (44-47) in SRO, the parameters of the symmetry energy 
   expansion (48-51) in SRO, and information about the effective masses of 
   nucleons for symmetric nuclear matter at nuclear saturation density, 
   and alpha particle mass, volume, and binding energy.
    
 \item \verbfile+symmetry.dat+ contains the symmetry energy of uniform
   nuclear matter as a function of density, see Equation (49) of SRO
   \cite{schneider:17}.

 \item \verbfile+surface_density.dat+ contains $y_i$, $T$ (in MeV), the
   densities $\log_{10}n_{ni}$, $\log_{10}n_{pi}$, $\log_{10}n_{no}$,
   and $\log_{10}n_{po}$ which are solutions to Equations (23) and (24) of
   SRO \cite{schneider:17} and the residue of the four equations.

 \item \verbfile+surface_tension.dat+ uses the density results given
   in \verbfile+surface_density.dat+ and provides $y_i$, $T$ (in MeV),
   the parameters $r_n$, $a_n$, and $a_t$ (in fm), which minimize the
   surface tension, the derivatives of $\sigma(y_i,T)$ w.r.t. $r_n$,
   $a_n$, and $a_p$, and the surface tension $\sigma(y_i,T)$ in
   MeV\,fm$^2$. Note that $r_p$ is set to zero by default. $r_t$ and
   $a_t$ and the surface tension are discussed around Equations (25-28)
   of SRO~\cite{schneider:17}.

 \item \verbfile+surface_tension_fit.dat+ contains $y_i$, $T$, the
   normalized surface tension $\sigma(y_i,T)/\sigma_s$, the fitting
   $\sigma_{\mathrm{fit}}(y_i,T)/\sigma_s$ given by Equations (19) and
   (20) of SRO~\cite{schneider:17}, and the ratio
   $\sigma_{\mathrm{fit}}(y_i,T)/\sigma(y_i,T)$.

 \item \verbfile+Surface_fit.dat+ contains the parameters $\sigma_s$,
   $\lambda$, $q$, and $p$ of the surface fit, Equations (19) and (20)
   in SRO~\cite{schneider:17}, as well as the surface properties $A_S$
   and $S_S$ in Equation (53) of SRO~\cite{schneider:17}.

 \item \verbfile+Critical_temperature_fit.dat+ provides the parameters
   $T_{c0}$, $a_c$, $b_c$, $c_c$, and $d_c$ that fit the critical
   temperature $T_c(y_i)$ for which two phases of nuclear matter with
   different densities and proton fractions coexist. See Equation (22) of
   SRO~\cite{schneider:17}.

\end{itemize}

Note that the files related to the surface fit will only be written if
explicit calculation of the surface fit takes place, \textit{i.e.}, if
\verbprm|fit_surface = .true.| in the \verbnml+SURFACE_PROPERTIES+
namelist.


The other output depends on whether \verbexec+test+ or \verbexec+main+
is run.  If \verbexec+test+ is run, output about the search for
solution, possible errors found, and ground state of the system, if
found, is printed to the standard output.

If \verbexec+main+ is run, then an \verb|HDF5| file is output, unless 
\verbprm|make_hdf5_table = .false.|. 




\subsubsection{Output of \texttt{SNA \textcolor{magenta}{test}}}\label{sssec:SNA_test_out}


The test output should be mostly self-explanatory.  However, some
error messages still need to be implemented.  Units are appropriate
combinations of MeV and fm unless otherwise stated explicitly.



\subsubsection{Output of \texttt{SNA \textcolor{magenta}{main}}}\label{sssec:SNA_main_out}


The \verbexec+main+ executable creates a directory with the name that
is specified in the \verbprm+output_directory+ parameter of namelist
\verbnml+TABLE_INPUT+.

If \texttt{\color{ForestGreen}write\_solutions\_to\_file = .true.} in
namelist \texttt{\color{cyan}TABLE\_INPUT}, then the code creates two
subdirectories \verbfile+SOL+ and \verbfile+NO_SOL+.  Although this
should work smoothly, some Fortran compilers have problems with the
system calls and are unable to create directories and subdirectories.
If that is the case, the user will need to create the necessary
directories manually. 

If \texttt{\color{ForestGreen}write\_solutions\_to\_file = .true.}, in
the \verbfile+SOL+ directory, a file with the solutions to the uniform
and non-uniform systems is written for each proton fraction. The
explanation of the columns of these files is given below.  In the
\verbfile+NO_SOL+ directory, a file for each proton fraction is
created and if neither a uniform nor a non-uniform solution is found
for a particular combinations of proton fraction, density, and
temperature, their values and possible errors encountered are written
to the file. The explanation of the columns is given below.


The files written in the \verbfile+output_directory/SOL/+ directory have
17 columns.  They are:

\begin{enumerate}
 \item[1.--3.] Density (fm$^{-3}$), temperature (MeV), and proton
   fraction $y$.

 \item[4.] Independent variable $y_p$ ($y_n$) that solves the uniform
   system if $y<0.50$ (if $y>0.50$). Set to zero if uniform solution
   not found or not searched for.

 \item[5.--7.] Independent variables $\zeta=[\log_{10}(n_{no}),
   \log_{10}(n_{po}), \log_{10}(u)]$ that solve the non-uniform
   system. Densities are assumed to be in fm$^{-3}$. Set to zero if
   non-uniform solution not found or not searched for.

 \item[8.--9.] Specific free energy in MeV
   for the uniform and non-uniform systems. Set to
   $10^{99}$ if solution not found or not searched for.

 \item[10.--11.] Dot product of residue for the equations solved for
   uniform and non-uniform systems. Set to $10^{10}$ if solution not
   found or not searched for.

 \item[12.--13.] Number of times tried to search for uniform and
   non-uniform solutions.

 \item[14.--15.] Logical (T or F) indicating whether uniform and
   non-uniform solutions were found.

 \item[16.--17.] Logical (T or F) indicating whether uniform and
   non-uniform solutions were searched for.

\end{enumerate}




The files written in the \verbfile+output_directory/NO_SOL/+ directory
have 9 columns.  They are:

\begin{enumerate}

\item[1.--3.] Density (fm$^{-3}$), temperature (MeV), and proton
  fraction $y$.

\item[4.--5.] Dot product of the residue of the equations solved for
  the uniform and non-uniform systems. Set to $10^{10}$ if solution
  not found or not searched for.

\item[6.--7.] Logical (T or F) indicating whether uniform and
  non-uniform solutions were found.

\item[8.--9.] Logical (T or F) indicating whether uniform and
  non-uniform solutions were searched for.

\end{enumerate}


\subsubsection{SNA HDF5 output}\label{sssec:SNA_HDF5}


The \verb|HDF5| output file, unless limited by the \verbnml|TABLE_OUTPUT| namelist discussed above, contains information about the following quantities.
\begin{enumerate}
 \item Scalar integer \verbprm+pointsrho+. Number of table points in
   logarithmic density.

 \item Scalar integer \verbprm+pointstemp+. Number of table points in
   logarithmic temperature.

 \item Scalar integer \verbprm+pointsye+. Number of table points in
   proton fraction.

 \item 1D Array
   \verbprm+logn+ with size \verbprm+pointsrho+. Contains the values for
   $\log_{10}n$, where $n$ is density in fm$^{-3}$.

 \item 1D Array \verbprm+logtemp+ with size
   \verbprm+pointstemp+. Contains the values for $\log_{10}T$, where
   $T$ is temperature in MeV.

 \item 1D Array \verbprm+ye+ with size \verbprm+pointsye+. Contains
   the values for the proton fraction.
   
 \item 3D Array \verbprm+p+ with size
   (\verbprm+pointsrho+,\verbprm+pointstemp+,\verbprm+pointsye+). Note
   that all 3D arrays described in the following have the same
   size. Contains the pressure $P$ in units of MeV\,fm$^{-3}$ for each
   point $(n,T,y)$ in the table.

 \item 3D Array \verbprm+s+. Contains the specific entropy in
   $k_B$\,baryon$^{-1}$.

 \item 3D Array \verbprm+e+. Contains the specific internal energy in
   MeV\,baryon$^{-1}$ with respect to the free neutron mass.

 \item 3D Array \verbprm+muh+. Contains $\hat\mu=\mu_n-\mu_p$ in MeV. 
   Note that $\hat\mu$ includes the neutron--proton mass difference.
   
 \item 3D Array \verbprm+mun+. Contains $\mu_n$ in MeV with respect to the
   free neutron mass.

 \item 3D Array \verbprm+mup+. Contains $\mu_p$ in MeV with respect to the
   free neutron mass.

 \item 3D Array \verbprm+dpdn+. Contains $dP/dn\vert_{T,y}$ in MeV.  

 \item 3D Array
   \verbprm+dsdn+. Contains $ds/dn\vert_{T,y}$ in $k_B$\,fm$^3$\,baryon$^{-1}$. 

 \item 3D Array \verbprm+dmudn+. Contains $d\mu_h/dn\vert_{T,y}$ in MeV\,fm$^3$. 

 \item 3D Array \verbprm+dpdT+. Contains $dP/dT\vert_{n,y}$ in fm$^{-3}$.

 \item 3D Array
   \verbprm+dsdT+. Contains $ds/dT\vert_{n,y}$ in $k_B$\,baryon$^{-1}$\,MeV$^{-1}$.

 \item 3D Array
   \verbprm+dmudT+. Contains $d\mu_h/dT\vert_{n,y}$. Dimensionless. 

 \item 3D Array \verbprm+dpdy+. Contains $dP/dy\vert_{n,T}$ in MeV\,fm$^{-3}$.

 \item 3D Array \verbprm+dsdy+. Contains $ds/dy\vert_{n,T}$ in  $k_B$\,baryon$^{-1}$.

 \item 3D Array
   \verbprm+dmudy+. Contains $d\mu_h/dy\vert_{n,T}$ in MeV. 

 \item 3D Array
   \verbprm+zbar+. Contains the charge number $Z$ of the representative
   heavy nucleus.

 \item 3D Array
   \verbprm+abar+. Contains the mass number $A$ of the representative
   heavy nucleus.

 \item 3D Array
   \verbprm+xn+. Contains the neutron mass fraction.

 \item 3D Array \verbprm+xp+. Contains the proton mass fraction.

 \item 3D Array \verbprm+xa+. Contains the alpha particle mass fraction.

 \item 3D Array \verbprm+xh+. Contains the heavy nuclei mass fraction.

 \item 3D Array \verbprm+u+. Contains the occupied volume fraction by
   heavy nuclei $u$. 

 \item 3D Array \verbprm+r+. Contains the occupied generalized radius
   of heavy nuclei $r$ in fm. 

 \item 3D Array
   \verbprm+meff_n+. Contains the effective mass of unbound neutrons $m^*_n$ in MeV.

 \item 3D Array \verbprm+meff_p+. Contains the effective mass of
   unbound protons $m^*_p$ in MeV.

\end{enumerate}

\smallskip
Note that the \texttt{SNA} code also includes additional datasets in
the \texttt{HDF5} file that contain the full \texttt{SNA} source code
and all inputs necessary to reproduce the results stored in the
\texttt{SNA} EOS table. See Section~\ref{sec:formaline} for details on
this feature.


\section{The Nuclear Statistical Equilibrium Code \texttt{NSE}}\label{sec:NSE}

The \texttt{NSE} code used to compute the EOS of an ensemble of nuclei
in nuclear statistical equilibrium resides in the directory
\verbfile|NSE|.


The \texttt{NSE} code works by computing the most favorable
composition of nuclei (the one with the lowest Helmholtz free energy)
for a user-specified ensemble of nuclei (and nucleons) and their
masses and partition functions. Details are discussed in the SRO paper
\cite{schneider:17}. NSE can be computed for a single point (the
\texttt{\color{magenta} test} case) or for a 3D space of points
(\texttt{\color{magenta} main} case) in proton fraction $y$,
temperature $T$, and density $n$.

\subsection{Compiling and Running the \texttt{NSE} Code}\label{ssec:NSE_dep}

Once all dependencies outlined in Section~\ref{sec:req} are installed
and the \texttt{make.inc} file is set up in the SRO EOS root
directory, compiling the \verb|NSE| code should be as simple as


\begin{verbatim}
 > cd NSE
 > make
\end{verbatim}


This will create two executables in the
\verbfile|NSE| directory: \verbexec|main| and \verbexec|test|.  
Their inputs and outputs are discussed in Section \ref{ssec:NSE_in}
and \ref{ssec:NSE_out}, respectively.

\textbf{IMPORTANT:} When solving for the EOS, the code will make use
of OpenMP parallelization (provided the compiler supports it and the
correct flags are set during compilation). It will use as many OpenMP threads as
specified by the system environment variable
\texttt{OMP\_NUM\_THREADS}. Make sure to check what this is set to and
adjust it, if you want to use less or more threads. We do not
recommend using more threads than available physical CPU cores on your
machine.

We tested the \texttt{NSE} code with various versions of the GNU
compiler suite and the Intel compiler suite.





\subsection{Input}\label{ssec:NSE_in}


The \verbexec|main| (\verbexec|test|) executable takes five (four)
input files.  Three of the input files are the same for both codes --
from the \texttt{\color{red}NSE} root directory:
\verbfile|input/list.in|, \verbfile|input/partition.in|, and
\verbfile|input/isotope_masses.in|.  The \verbexec|main| executable 
takes two other inputs: the \verbfile|input/output.in|
and \verbfile|input/space.in| files, which contain,
respectively, details on the output to be written by the code and the
space in proton fraction, density, and temperature to be covered by
the output \texttt{NSE} table.  The \verbexec|test| executable reads
the \verbfile|input/test.in| file, which contains a single point in
proton fraction, density, and temperature to compute the NSE EOS.  We
discuss these input files in detail below.




\subsubsection{The Mass Table and Partition Function Input}
\label{sssec:mass_table.in}



The mass table input,
\verbfile|input/isotopes.in|, contains information about the names of
available nuclides and their proton, neutron, and mass number, as well
as their mass excess and binding energies.  \textbf{We advise the user
  to NOT modify this file.}  Information about the nuclide partition
functions contained in the \verbfile|input/partition.in| file.
\textbf{Again, we do NOT recommend the user to modify this file}.  The
data in both of these files are obtained from the file
\verbfile+chem/data/isotopes.data+ of the open-source MESA stellar
evolution code \cite{MESA}, revision r7503, which contains information
for 7851 nuclides.  The mass table contains 6 columns with nuclide
properties: a short name descriptor, mass number, proton number
(charge), neutron number, mass excess w.r.t.\ $^{12}$C, and the nuclear
binding energy.  The excess mass, binding energies and partition
function are from the JINA REACLIB library
\cite{cyburt:10}. Here the binding energies are obtained
  w.r.t.\ symmetric nuclear matter. However, to be consistent with the
\verbexec+SNA+ code, we subtract the neutron-proton mass
difference from the resulting free energies.


\subsubsection{Input Nuclides}\label{sssec:NSE_isotopes}


To construct the \texttt{NSE} table, one must provide a list of
nuclides in file
\verbfile|input/list.in|. This file is actually a symbolic link
that points to a nuclide list file in the  \verbfile|tables|
directory (see below).

When making a custom \verbfile|input/list.in| file, an important
restriction to keep in mind is that the chosen nuclides must be
present in the
\verbfile|input/isotopes.in| and \verbfile|input/partition.in| files.
For obvious reasons, we recommend protons, neutrons, and alpha
particles to be among the nuclides included when putting together
\verbfile|input/list.in|.  The \verbfile|input/list.in| file must have
four columns: the nuclide name, which should match the name in the
\verbfile|input/isotopes.in| file, its mass number, proton (charge)
number, and neutron number.  \textbf{Because of the way the code is
  structured, the nuclide list must be ordered by increasing proton
  (charge) number first and second by increasing neutron number.}  We
provide in the
\verbfile|tables| directory the nuclide lists used in the SRO paper 
and some other nuclide lists inspired by those used by MESA~\cite{MESA}.




\subsubsection{NSE test input}\label{sssec:NSE_test_in}


The \verbexec+test+ executable takes as input the namelist
\verbnml+input_test_list+, which is part of the
\verbfile+input/test.in+ file.  This namelist determines the point in
proton fraction, density, and temperature where to calculate the NSE
EOS.

{\color{cyan}
\begin{verbatim}
&input_test_list
! point in phase space
proton_fraction =  30.0D-02, ! proton fraction for test
temperature     =  1.00D-00, ! temperature in MeV for test
density         =  1.00D-03, ! density in fm^-3 for test
/
\end{verbatim}}



\subsubsection{NSE Table Ranges and Resolution Input File}\label{sssec:NSE_phase_space_in}


As in the \texttt{SNA} code, the \verbexec+main+ \texttt{NSE}
executable also takes as input a namelist
\verbnml+space_list+ that is in the file
\verbfile+input/space.in+ (path relative to the \texttt{NSE} root directory; 
this file is distinct from its \texttt{SNA} counterpart).  This namelist
determines the range and spacing in $\log_{10}n$ (density $n$ given in
fm$^{-3}$), $\log_{10}T$ (temperature $T$ given in MeV), and proton
fraction $y$ that is covered by the output \texttt{NSE} table. The
example below should be self-explanatory.  Note that there are two
methods for determining the spacing for each of the three variables
and that we comment out the superseding option.  We do not try to
obtain the NSE EOS at densities larger than nuclear saturation density
$n_0\simeq0.16\,\mathrm{fm}^{-3}$ since we expect matter to be a
uniform gas of protons and neutrons there.  In fact, the NSE EOS if
likely unreliable for densities of $\gtrsim n_0/10$ since exotic
nuclei dominate in this region and nuclear interactions play an
important role and are not included in our NSE framework.


{\color{cyan}
\begin{verbatim}
&space_list
yp_min      =  0.00499D0,       ! minimum proton fraction
yp_max      =  0.66499D0,       ! maximum proton fraction
steps_in_Yp =  67,              ! steps in Yp every 0.01
! Yp_spacing = 0.01D0,          ! spacing in Yp 
! (if used Yp_spacing supersedes option for steps_per_hundredth_in_yp)

log10n_min  = -13.20d0,         ! log10 of minimum density in fm^-3
log10n_max  =  -1.0d0,          ! log10 of maximum density in fm^-3
steps_per_decade_in_n  =  30,   ! steps per decade in density
! log10n_spacing = 0.03333333d0,! spacing in log10(n) 
! (if used log10n_spacing supersedes option for steps_per_decade_in_n)

log10T_min  =  -3.1d0,          ! log10 of minimum density in fm^-3
log10T_max  =   2.5d0,          ! log10 of maximum density in fm^-3
steps_per_decade_in_T  = 30,    ! steps per decade in density
! log10T_spacing = 0.03333333d0,! spacing in log10(T) 
! (if used log10T_spacing supersedes option for steps_per_decade_in_T)
/
\end{verbatim}}

The \verbfile|input/output.in| file contains the \verbnml|TABLE_INPUT|
namelist. Similarly to the \verbexec+SNA+ case, this namelist contains
information about the type of output files to write, their names, and
the directory they are written to. Below is an example of the
\verbnml|TABLE_INPUT| namelist.

{\color{cyan}
\begin{verbatim}
&TABLE_INPUT
  make_hdf5_table = .true.,
  write_solutions_to_file = .false., 
  filename_base   = "NSE_0023",  ! base for HDF5 table name
  output_directory= "NSE_0023",  ! directory for ascii output files
/
\end{verbatim}}
The logical parameters in the namelist are set by default to the
values given above and only need to be explicitly set if changed.
Note that we have chosen the \texttt{0023} in the filename to
correspond to the default number of NSE nuclides used. Of course,
larger NSE networks are possible and should be considered.

The \verbprm|write_solutions_to_file| variable determines whether 
to write the neutron and proton chemical potentials, $\tilde\mu_n$ 
and $\tilde\mu_p$, respectively, that determine the nuclear 
statistical equilibrium and other debugging information for each 
point in density, temperature and proton fraction to an ASCII file. 
By default it is set to \verbprm|false|. 


The parameter \verbprm|make_hdf5_table| sets whether to write an 
\verb|HDF5| table in the \verbfile|output_directory| directory. 
The table name starts with \verbprm|filename_base| followed by 
information about the table resolution in $\log_{10}$(density), 
$\log_{10}$(temperature), and proton fraction. Next is information on 
the git revision. If the code to make the table was compiled from a 
repository with local changes, the short git hash is prepended with a 
captial ``M''. If the repository is all fully committed, the short git 
hash has no prefix. Finally, the name contains the date the table was
generated. For example, a table name could look like 
this: \verbfile|NSE_0023rho367_temp169_ye67_gitMd089cf6_20170624.h5|.  
Note that the table is only written after all solutions have been
computed. Also note that we are including the full source code and all
inputs needed to generate the table. See Section~\ref{sec:formaline}
for details on this.


The details of the contents (including units) of the \verb|HDF5| output file 
are discussed in Section~\ref{sssec:NSE_main_out}.


\subsection{\texttt{NSE} Output}\label{ssec:NSE_out}


\subsubsection{Output of Test}\label{sssec:SNA_test_out}


The test output should be mostly self-explanatory.  However, some
error messages still need to be implemented.  Units are appropriate
combinations of MeV and fm unless otherwise stated explicitly.


\subsubsection{Output of main}\label{sssec:NSE_main_out}


The \verbexec+main+ executable creates a directory with the name that
is specified in the \verbprm+output_directory+ parameter of namelist
\verbnml+TABLE_INPUT+.

The following applies only if \verbprm|write_solutions_to_file = .true.| in the
\texttt{\color{cyan}TABLE\_INPUT} namelist:

In the \verbprm+output_directory+, the code creates three
subdirectories \verbfile+SOL+, \verbfile+NO_SOL+, and
\verbfile+ERROR+.  Although all of these directories should
automatically be created by the code, some Fortran compilers have
problems with the system calls and are unable to create directories
and subdirectories.  If that is the case, the user will need to create
the necessary directories manually.


In the \verbfile+SOL+ directory, files for each proton fraction computed 
containing the neutron and proton chemical potentials ($\tilde\mu_n$ and 
$\tilde\mu_p$) that determine the composition of isotopes given in nuclear 
statistical equilibrium is given. 
In the \verbfile+NO_SOL+ directory, a file for each proton fraction is 
created and if the solution found is not within the desired tolerance. 
In the \verbfile+ERROR+ directory, a file for each proton fraction is 
created and an error message is written to the corresponding file if there 
is a problem solving the system of equations. 
An explanation of the columns in the files of each of these folders is 
given below.


The files written in the \verbfile+output_directory/SOL/+ directory have
13 columns.  They are:

\begin{enumerate}
 \item[1.--3.] Density (fm$^{-3}$), temperature (MeV), and proton
   fraction $y$.
 \item[4.--5.] Neutron and proton chemical potentials, 
   $\tilde\mu_n$ and $\tilde\mu_p$. Set to zero if a solution is not found. 
 \item[6.] Accuracy of solution defined by $\vert1-x_n-x_p-x_\alpha-x_l-x_h\vert$.
  \begin{itemize}
   \item $x_n$ is the mass fraction of neutrons.
   \item $x_p$ is the mass fraction of protons.
   \item $x_\alpha$ is the mass fraction of alpha particles. 
   \item $x_l$ is the mass fraction of light nuclei. 
         Light nuclei are defined as nuclei with proton number $Z\leq6$,
         not including neutrons, protons, and alpha particles.
   \item $x_h$ is the mass fraction of heavy nuclei. Heavy nuclei areas
         defined as nuclei with proton number $Z>6$.
  \end{itemize}
 \item[7.--11.] The mass fractions $x_n$, $x_p$, $x_\alpha$, $x_l$, $x_h$.
                Set to $10^{-99}$ if lower than that. 
 \item[12.] Number of calls to the function \texttt{nr\_iterate} that solves
            the system of equations. 
 \item[13.] Number of calls to the function \texttt{reorder} that orders
            nuclei by their abundances. 
\end{enumerate}


The files written in the \verbfile+output_directory/NO_SOL/+ directory
have the same columns as the files in \verbfile+output_directory/SOL/+.
The difference is that a line is only written if the solution is not found 
or not whithin the accepted tolerance, 
$\vert1-x_n-x_p-x_\alpha-x_l-x_h\vert < 10^{-6}$.


The files written in the
\verbfile+output_directory/ERROR/+ directory have four columns. The
columns are only written in case the subroutine \texttt{local\_nse} is
unable to find a combination of nuclides in NSE. The columns are

\begin{enumerate}
 \item[1.--3.] Density (fm$^{-3}$), temperature (MeV), and proton fraction $y$. 
 \item[4.] Error type:
 \begin{itemize}
  \item 1, if the number of iterations trying to compute the solution
    is larger than the allowed maximum, 600.
  \item 2, if the Jacobian of the system of equations is singular. 
 \end{itemize}
\end{enumerate}


\subsubsection{NSE HDF5 output}\label{sssec:NSE_HDF5}

The \verb|HDF5| output file contains the following datasets:

\begin{enumerate}
 \item Scalar integer \verbprm+pointsrho+. Number of table points in
   logarithmic density.

 \item Scalar integer \verbprm+pointstemp+. Number of table points in
   logarithmic temperature.

 \item Scalar integer \verbprm+pointsye+. Number of table points in
   proton fraction.

 \item 1D Array
   \verbprm+logn+ with size \verbprm+pointsrho+. Contains the values for
   $\log_{10}n$, where $n$ is density in fm$^{-3}$.

 \item 1D Array \verbprm+logtemp+ with size
   \verbprm+pointstemp+. Contains the values for $\log_{10}T$, where
   $T$ is temperature in MeV.

 \item 1D Array \verbprm+ye+ with size \verbprm+pointsye+. Contains
   the values for the proton fraction.
   

 \item 3D Array \verbprm+p+ with size
   (\verbprm+pointsrho+,\verbprm+pointstemp+,\verbprm+pointsye+). Note
   that all 3D arrays described in the following have the same
   size. Contains the pressure $P$ in units of MeV\,fm$^{-3}$ for each
   point $(n,T,y)$ in the table.

 \item 3D Array \verbprm+s+. Contains the specific entropy in
   $k_B$\,baryon$^{-1}$.

 \item 3D Array \verbprm+e+. Contains the specific internal energy in
   MeV\,baryon$^{-1}$.

 \item 3D Array \verbprm+muh+. Contains $\hat\mu=\mu_n-\mu_p$ in MeV. 
   Note that here $\mu_n$ and $\mu_p$ are obtained with respect to the neutron vacuum mass. 
   
 \item 3D Array \verbprm+mun+. Contains $\mu_n$ in MeV. 

 \item 3D Array \verbprm+mup+. Contains $\mu_p$ in MeV.

 \item 3D Array \verbprm+dpdn+. Contains $dP/dn\vert_{T,y}$ in MeV.  

 \item 3D Array \verbprm+dsdn+. Contains $ds/dn\vert_{T,y}$ in
 $k_B$\,fm$^3$\,baryon$^{-1}$. 

 \item 3D Array \verbprm+dmudn+. Contains $d\mu_h/dn\vert_{T,y}$ in MeV\,fm$^3$. 

 \item 3D Array \verbprm+dpdT+. Contains $dP/dT\vert_{n,y}$ in fm$^{-3}$.

 \item 3D Array \verbprm+dsdT+. 
   Contains $ds/dT\vert_{n,y}$ in $k_B$\,baryon$^{-1}$\,MeV$^{-1}$. 

 \item 3D Array \verbprm+dmudT+. Contains $d\mu_h/dT\vert_{n,y}$. Dimensionless. 

 \item 3D Array \verbprm+dpdy+. Contains $dP/dy\vert_{n,T}$ in MeV\,fm$^{-3}$ 

 \item 3D Array \verbprm+dsdy+. Contains $ds/dy\vert_{n,T}$ in  $k_B$\,baryon$^{-1}$. 

 \item 3D Array \verbprm+dmudy+. Contains $d\mu_h/dy\vert_{n,T}$ in MeV. 

 \item 3D Array \verbprm+xn+. Contains the neutron mass fraction. 

 \item 3D Array \verbprm+xp+. Contains the proton mass fraction.

 \item 3D Array \verbprm+xa+. Contains the alpha particle mass fraction.

 \item 3D Array \verbprm+xh+. Contains the heavy nuclei mass fraction.

 \item 3D Array \verbprm+xl+. Contains the heavy nuclei mass fraction.

 \item 3D Array \verbprm+zh+. Contains the charge number $Z_h$ 
   averaged over the present heavy nuclei.

 \item 3D Array \verbprm+ah+. Contains the mass number $A_h$ 
   averaged over the present heavy nuclei.

 \item 3D Array \verbprm+zl+. Contains the charge number $Z_l$ 
   averaged over the present light nuclei.

 \item 3D Array \verbprm+al+. Contains the mass number $A_l$ 
   averaged over the present light nuclei.

\end{enumerate}

\smallskip
Note that the \texttt{NSE} code also includes additional datasets in
the \texttt{HDF5} file that contain the full \texttt{NSE} source code
and all inputs necessary to reproduce the results stored in the
\texttt{NSE} EOS table. See Section~\ref{sec:formaline} for details on
this feature.


\section{The Code to Merge Tables: \texttt{MERGE}}
\label{sec:merge}


The \texttt{MERGE} code (residing in the \texttt{MERGE} subdirectory) 
is used to combine the HDF5 tables output by the \texttt{SNA} and 
\texttt{NSE} codes into a new table using the procedure described
in Section VII.A. of SRO~\cite{schneider:17}. The \texttt{MERGE} code
also adds photon, electron, and positron contributions to the EOS.


\subsection{Compiling and Running the \texttt{MERGE} Code}


Make sure your system satisfies the requirements outline in
\S\ref{sec:req} and that your \texttt{make.inc} file is set up
correctly. Given that, compiling the \verb|MERGE| code should now be
as simple as typing in your terminal from the root directory of the
SRO EOS:

\begin{verbatim}
 > cd MERGE
 > make
\end{verbatim}


This will create a single executable in the \verbfile|MERGE| 
directory called \verbexec|merge_eos|.  Its input and output are
discussed in Sections \ref{ssec:MERGE_in} and \ref{ssec:MERGE_out},
respectively. 


\textbf{IMPORTANT:} When creating the combined EOS table, the code
will make use of OpenMP parallelization (provided the compiler
supports it and the correct flag is set during compilation). It will
use as many OpenMP threads as specified by the system environment
variable \texttt{OMP\_NUM\_THREADS}. Make sure to check what this is
set to and adjust it, if you want to use less or more threads. We do
not recommend using more threads than available physical CPU cores on
your machine.


\subsection{Input}
\label{ssec:MERGE_in}


The \texttt{\textcolor{magenta}{merge\_eos}} executable takes three input 
files, \verbfile|input/tables.in|, \verbfile|input/space.in|, and 
\verbfile|input/transition.in|. They are discussed next.


\subsubsection{The \texttt{\textcolor{magenta}{input/tables.in}} file}
\label{sssec:Merge_tables}

\verbfile|input/tables.in| contains one namelist, \verbnml|tables|.
An example is given below. 


{\color{cyan}
\begin{verbatim}
&tables
merge_base = 'SKRA_0023',
sna_eos = '../SNA/EOS/SKRA/SKRA_rho427_temp169_ye67_gitMd089cf6_20170624.h5',
nse_eos = '../NSE/NSE_0023/NSE_0023_rho367_temp169_ye67_gitMd089cf6_20170624.h5',
only_sna = .FALSE.,
only_nse = .FALSE.,
/
\end{verbatim}}

This namelist \verbnml|tables| contains five parameters.  Three
parameters are character strings: \verbprm+merge_base+ is a base for the 
file name of the final table (we recommend including the Skyrme 
parametrization and the number of NSE nuclides [if any]),
\verbprm+sna_eos+ is the path to the SNA table,
\verbprm+nse_eos+ is the path to the NSE table.  

The two remaining parameters are logical,
\verbprm+only_sna+ and \verbprm+only_nse+ determine whether the final
HDF5 table is obtained only in the SNA or only in the NSE
approximation.  Note that both options cannot be set to
\texttt{\textcolor{ForestGreen}{.true.}} simultaneously.

Note that in the above case, the resulting output table will be named
\verbfile|SKRA_0023_rho415_temp163_ye66_gitMd089cf6_20170624.h5|. Note
that the \texttt{M} behind \texttt{git} indicates that the table was
generated from a not fully committed git repository. The 7 following
characters are the short git hash. The final set of numbers is the date
the table was created at in \texttt{YYYY-MM-DD} format.




\subsubsection{The \texttt{\textcolor{magenta}{input/space.in}} file}
\label{sssec:MERGE_phase}


As the \verb+SNA+ and \verb+NSE+ codes, the \verb+MERGE+ code also
takes as input a file named
\verbfile|input/space.in|.  This file has a namelist
\verbnml+space_list+ that specifies the ranges and resolutions in
density, temperature, and proton fraction for the final merged table.
An example is shown below.

{\color{cyan}
\begin{verbatim}
&space_list
yp_min      =  0.005D0,       ! minimum proton fraction
yp_max      =  0.655D0,       ! maximum proton fraction
steps_in_Yp =  66,              ! steps in Yp every 0.01
! Yp_spacing = 0.01D0,          ! spacing in Yp 
! (if used Yp_spacing supersedes option for steps_per_hundredth_in_yp)

log10n_min  = -13.0d0,          ! log10 of minimum density in fm^-3
log10n_max  =   0.8d0,          ! log10 of maximum density in fm^-3
steps_per_decade_in_n  =  30,   ! steps per decade in density
! log10n_spacing = 0.03333333d0,! spacing in log10(n) 
! (if used log10n_spacing supersedes option for steps_per_decade_in_n)

log10T_min  =  -3.0d0,          ! log10 of minimum density in fm^-3
log10T_max  =   2.4d0,          ! log10 of maximum density in fm^-3
steps_per_decade_in_T  = 30,    ! steps per decade in density
! log10T_spacing = 0.03333333d0,! spacing in log10(T) 
! (supersedes log10T_spacing option for steps_per_decade_in_n)
/
\end{verbatim}}

Before getting into the details, we want to point out the following:

\medskip
\begin{framed}\textbf{IMPORTANT:} 
  The \texttt{MERGE} code interpolates linearly in $\log_{10} n$,
  $\log_{10} T$ and $y$ to fill points that are not exactly aligned
  with entries in the SNA and NSE tables. Because of this, great care
  must be taken to set up density, temperature, and proton fraction
  grids for SNA, NSE, and merged tables. We recommend to \emph{always}
  make SNA and NSE tables that are higher resolution than the desired
  resolution for the final merged table. For the merged tables we make
  available at \url{https://stellarcollapse.org/SROEOS}, we used SNA
  and NSE tables that have twice the resolution of the merged tables.
\end{framed}

While in the \verb+SNA+ and \verb+NSE+ codes the specified ranges are
limited by physical significance and computational limitations, for
the \verb+MERGE+ code the ranges are limited by the ranges of the SNA
and NSE tables being merged.

If the option \verbprm+only_sna+ (\verbprm+only_nse+) is set to
\texttt{\textcolor{ForestGreen}{.true.}}, then the space in density,
temperature, and proton fraction space given in the namelist
\verbnml+space_list+ for the \texttt{MERGE} code should be completely inside
the space used by the \verbnml+space_list+ namelist used to generate
the SNA (NSE) table.

In the standard case that SNA and NSE tables are merged, the range
requirements change slightly:

\begin{enumerate}[label=(\arabic*)]

  \item Our merging procedure is only a function of density. Hence,
    \verbprm+yp_min+ and \verbprm+log10T_min+ in \verb+MERGE+ should
    be larger than or equal to the largest value of these parameters
    used for the SNA and/or NSE tables.

\item Similarly, the parameters \verbprm+yp_max+ and
  \verbprm+log10T_max+ used in the \verb+MERGE+ code should be smaller than or equal to the smallest value of these parameters
  used for the SNA and/or NSE tables.

\item We always assume that the NSE table is at a lower density than
  the SNA table. Hence, the requirements for the density are such that
  \verbprm+log10n_min+ for \verb+MERGE+ is larger than or equal to the value of that parameter
  used to generate the NSE table. Similarly,
  \verbprm+log10n_max+ used for \verb+MERGE+ should be smaller than or equal to the value of this parameter
  used to generate the SNA table.

\end{enumerate}


We discuss in Section \ref{sssec:Merge_transition} how the maximum
density of the NSE table and minimum density in the SNA table are
limited by the transition density, its width, and the merge tolerance.


\subsubsection{The \texttt{\textcolor{magenta}{input/transition.in}} file}
\label{sssec:Merge_transition}

The file
\verbfile+input/transition.in+ contains a namelist \verbnml+transition+ that specifies
the transition density $n_t$ from SNA to NSE and the transition width
$n_\delta$, see Section VII.A.\ of SRO~\cite{schneider:17}. We also
define a tolerance $\epsilon_t$ used for the merge function that will
be discussed in the following.  An example of the \verbnml+transition+
namelist is shown below.


{\color{cyan}
\begin{verbatim}
&transition
  n_transition = -4.00d0,   ! Log10(transition density in fm^-3)
  n_delta      =  0.33d0,   ! transition thickness parameter
  n_tolerance  =  1.d-04,   ! tolerance in transition function
/
\end{verbatim}}


The function we use to merge the SNA and NSE Helmholtz free energies,
\begin{equation}
F_{\textrm{MIX}}=\chi(n) F_{\textrm{SNA}} + [1-\chi(n)] F_{\textrm{NSE}}\,,
\end{equation}
depends on the function
\begin{equation}
 \chi(n) = \frac{1}{2}\left[1+\tanh\left(\frac{\log_{10}(n)-\log_{10}(n_t)}{n_\delta}\right)\right]\,.
\end{equation}
Since $\chi(n)$ only reaches 0 or 1 asymptotically.
Hence, to speed up the \verb+MERGE+ code and avoid having to determine 
the SNA and NSE EOS contributions in regions where they are very small, 
we limit the range to compute $F_{\textrm{MIX}}$ to densities $n_-<n<n_+$
where 
\begin{equation}
 \log_{10}n_\pm = n_t \mp n_\delta \tan^{-1}\left(2\epsilon_t-1\right)\,.
\end{equation}
Thus, when $\chi(n)\leq\epsilon_t$ the EOS is only given by the NSE EOS, 
while if $\chi(n)\geq1-\epsilon_t$ the EOS is given by the SNA EOS. 


Given these constraints, when merging an SNA with an NSE EOS, the
choices of $n_t$, $n_\delta$, and $\epsilon_t$ should be such that
$\log_{10}n_+<$
\verbprm+log10n_max+ for NSE table and $\log_{10}n_->$ \verbprm+log10n_min+
for the SNA table.  Even though the \verb+MERGE+ code does warn the
user of any issues with the parameters chosen in the
\verbnml+transition+ namelist, it is up to the user to chose
transition parameters that satisfy the constraints discussed.  We
refer the user to Section VII.A.\ of SRO~\cite{schneider:17} and
recommend that the user carefully test the effects on their
application problem of variations of the above transition parameters.



\subsection{Output}
\label{ssec:MERGE_out}

The \verbexec+merge_eos+ executable creates an HDF5 table that
combines the \verb+SNA+ and \verb+NSE+ tables with the electron and
photon EOSs. The units now are, in most cases, in cgs units. The
quantities output in the final table are listed below.

\begin{enumerate}
 \item Scalar integer \verbprm+pointsrho+. Number of table points in
   logarithmic density. 

 \item Scalar integer \verbprm+pointstemp+. Number of table points in
   logarithmic temperature. 

 \item Scalar integer \verbprm+pointsye+. Number of table points in
   proton fraction. 

 \item 1D Array
   \verbprm+logrho+ with size \verbprm+pointsrho+. Contains the values
   for $\log_{10}\rho$, where $\rho$ is the baryon density in
   g\,cm$^{-3}$.  The single baryon rest mass is the free neutron
   mass. 
   
 \item 1D Array \verbprm+logtemp+ with size
   \verbprm+pointstemp+. Contains the values for $\log_{10}T$, where
   $T$ is temperature in MeV. 

 \item 1D Array \verbprm+ye+ with size \verbprm+pointsye+. Contains
   the values for the proton fraction.
   
 \item 3D Array \verbprm+logpress+ with size
   (\verbprm+pointsrho+,\verbprm+pointstemp+,\verbprm+pointsye+). 
   Contains $\log_{10}P$ for each point $(\rho,T,y)$ in the table
   where $P$ is the pressure in units of dyne\,cm$^{-2}$. Note that 
   all 3D arrays described in the following have the same size. 
   
 \item 3D Array \verbprm+entropy+. Contains the specific entropy in
   $k_B$\,baryon$^{-1}$. 

 \item 3D Array
   \verbprm+logenergy+. Contains $\log_{10}(\epsilon+\epsilon_0)$
   where $\epsilon$ is the specific internal energy in erg\,g$^{-1}$
   with respect to the free neutron mass and
   $\epsilon_0=20$\,MeV$=1.91313\times10^{19}$\,erg\,g$^{-1}$ is an
   energy shift set to guarantee $\epsilon+\epsilon_0>0$.

 \item Scalar integer \verbprm+energy_shift+. Contains the energy shift
   $\epsilon_0=20$\,MeV$=1.91313 \times 10^{19}$\,erg\,g$^{-1}$ in cgs units.
   
 \item 3D Array \verbprm+muhat+. Contains $\hat\mu=\mu_n-\mu_p$ in MeV. 
   Note that $\hat\mu$ includes the neutron--proton mass difference. 
   
 \item 3D Array \verbprm+mu_n+. Contains the neutron chemical potential $\mu_n$ 
   in MeV with respect to the free neutron mass.

 \item 3D Array \verbprm+mu_p+. Contains the proton chemical potential $\mu_p$ 
   in MeV with respect to the free neutron mass.
   
 \item 3D Array \verbprm+mu_e+. Contains the electron chemical potential $\mu_e$ 
   in MeV including the electron rest mass. 
   
 \item 3D Array \verbprm+munu+. Contains the neutrino chemical potential 
   $\mu_\nu=\mu_n-\mu_p+\mu_e$ in MeV. 
   
 \item 3D Array \verbprm+dedt+. Contains $d\epsilon/dT\vert_{\rho,y}$ in units of
   erg\,g$^{-1}$\,K$^{-1}$. 

 \item 3D Array \verbprm+dpdrhoe+. Contains $dP/d\rho\vert_{\epsilon,y}$ in units of
   cm$^2$\,s$^{-2}$.

  \item 3D Array \verbprm+dpderho+. Contains $dP/d\epsilon\vert_{\rho,y}$ in units of
    g\,cm$^{-3}$.
   
 \item 3D Array \verbprm+gamma+. Contains the dimensionless adiabatic index 
   $\Gamma=d\log P/d\log \rho|_s$.
 
 \item 3D Array \verbprm+cs2+. Contains the speed of sound squared in units of 
   cm$^2$\,s$^{-2}$.
 
 \item 3D Array \verbprm+zbar+. Contains the average charge number $Z_h$ of 
   heavy nuclei. For the NSE EOS only isotopes with $Z>6$ are considered ``heavy''.
   See the discussion in Section \ref{sssec:NSE_main_out}.

 \item 3D Array \verbprm+abar+. Contains the average mass number $A_h$ of 
   heavy nuclei. For the NSE EOS only isotopes with $Z>6$ are considered ``heavy''.
   See the discussion in Section \ref{sssec:NSE_main_out}.

 \item 3D Array \verbprm+zlbar+. Contains the average charge number $Z_l$ of 
   light nuclei. For the NSE EOS only isotopes with $Z\leq6$ excluding protons, 
   neutrons, and alpha particles are considered ``light''. 
   See the discussion in Section \ref{sssec:NSE_main_out}. 
   
 \item 3D Array \verbprm+albar+. Contains the average charge number $A_l$ of 
   light nuclei. For the NSE EOS only isotopes with $Z\leq6$ excluding protons, 
   neutrons, and alpha particles are considered ``light''. 
   See the discussion in Section \ref{sssec:NSE_main_out}. 
   
 \item 3D Array \verbprm+xn+. Contains the neutron mass fraction. 

 \item 3D Array \verbprm+xp+. Contains the proton mass fraction. 

 \item 3D Array \verbprm+xa+. Contains the alpha particle mass fraction. 

 \item 3D Array \verbprm+xh+. Contains the heavy nuclei mass fraction. 
 
 \item 3D Array \verbprm+xl+. Contains the light nuclei mass fraction. 

 \item 3D Array \verbprm+u+. Contains the occupied volume fraction by
   heavy nuclei $u$. Note that as of now we do not use an excluded volume for 
   the nuclei in the NSE EOS and, thus, its contribution is zero. 

 \item 3D Array \verbprm+r+. Contains the occupied generalized radius
   of heavy nuclei $r$ in fm. Set to zero for the NSE EOS. 

 \item 3D Array \verbprm+meff_n+. Contains the effective mass of unbound 
   neutrons $m^*_n$ in MeV. Set to zero for the NSE EOS. 

 \item 3D Array \verbprm+meff_p+. Contains the effective mass of
   unbound protons $m^*_p$ in MeV. Set to zero for the NSE EOS. 

\end{enumerate}

\smallskip
Note that the \texttt{MERGE} code also includes additional datasets in
the \texttt{HDF5} file that contain the full \texttt{MERGE} source
code and all inputs necessary to reproduce the results stored in the
merged EOS table. It also includes all source code and inputs needed
to reproduce the NSE and SNA parts of the EOS. See
Section~\ref{sec:formaline} for details on this feature.


\section{EOS Driver and Interpolation Codes}

The HDF5 EOS tables generated by the SRO EOS code use the same overall
format and the same physical units for thermodynamic quantities as the
legacy EOS tables available from
\url{https://stellarcollpase.org/equationofstate} in the format
originally chosen by O'Connor \& Ott (2010,
\cite{oconnor:10}). Routines that can handle the legacy EOS tables
will be able to handle SRO tables.

That said, a few caveats need to be kept in mind:

\begin{enumerate}[label={(\arabic*)}]
\item The SRO EOS provides additional variables that are not in the
  legacy O'Connor \& Ott tables (see
  Section~\ref{ssec:MERGE_out}). Readers and interpolation routines
  need adjustment in order to be able to provide these additional
  variables.
  
\item The legacy O'Connor \& Ott tables include the speed of sound
  without relativistic corrections. However, the new SRO EOS tables
  include it with these corrections. SRO tables include an integer
  flag called \texttt{have\_rel\_cs2} in the HDF5 file that, if set to
  \texttt{1} (as is the default for SRO tables), indicates that the
  speed of sound is already relativistic and does not need to be
  corrected by the application code. The legacy tables do not have
  such a dataset, so application code can judge from its absence that
  the speed of sound needs correction.

  Specifically, we are talking about the following:

  \begin{equation}
    c_s^2 = \frac{c^2}{h} \frac{dP}{d\rho}\bigg|_s\,\,,
   \end{equation}
  with
  $h = c^2 + \epsilon + p/\rho$ and $\epsilon$ being the specific
  internal energy and $c$ being the speed of light.

% \item three
\end{enumerate}


We provide two different example codes for using SRO EOS tables. They
implement linear interpolation (and extrapolation) and root finding
for inverting $\epsilon(\rho,T,Y_e)$ to find the temperature at fixed
$\rho, Y_e$ after the update of the internal energy by the
(radiation-)hydrodynamics scheme.

\subsection{\texttt{Fortran 90} Interface/Driver/Interpolator}


This code is written in \texttt{Fortran 90} and is a somewhat updated
version of the code provided at
\url{https://stellarcollapse.org/equationofstate}.

The \texttt{Fortran 90} code is available as open source via a
\texttt{git} repository hosted on \texttt{bitbucket.org}. To obtain
the code, execute

\noindent \texttt{git clone https://bitbucket.org/zelmani/eosdriverfortran}

We do not recommend using the \texttt{Fortran 90} code in simulations.
Much of it dates back more than a decade. It is inefficient and a
display of rather poor software engineering. (But it works).

\textbf{Note that the current versions of this EOS
  interface/driver/interpolation implementation supports handling the
  relativistic speed of sound corrections as described above.}


\subsection{C/C++ Interface/Driver/Interpolator}

For high-performance applications, we recommend the more recent and much
better designed C/C++ routines available at


\noindent \texttt{git clone https://bitbucket.org/zelmani/eosdrivercxx}

These routines have been optimized for speed (including vectorization)
and perform up to a factor of three faster than the legacy
\texttt{Fortran 90} routines.

\textbf{Note that the current versions of this EOS
  interface/driver/interpolation implementation supports handling the
  relativistic speed of sound corrections as described above.}

\section{Code and Data Provenance: \texttt{Formaline}}
\label{sec:formaline}

Provenance is a word used in computational science (and in some other
research areas). It describes the full history of a (computational)
scientific result. \textbf{Knowing the provenance of a simulation
  result or dataset makes it fully reproducible}. The provenance of a
calculation does not only include the released open-source code, but
also the full information on the exact source code that was used,
perhaps even the way it was compiled and where it was run, the
parameters that were set for the calculation, and all initial
conditions and other ingredients that were used in whatever form (and
also their respective provenances).

Historically, EOS tables for core-collapse supernova and neutron star
merger simulations were stored in proprietary (binary) formats that
did not provide the user with information on the table format
and on how the table was generated, which code and parameters were
used. This made it often impossible to reproduce how a given table was
generated and, in the worst case, made it impossible to use the table
if the format was no longer known. \texttt{HDF5} solves the latter
problem, but by itself does not provide information on how the table
was generated.


The SRO EOS code uses the \texttt{Formaline} approach to boundle
source code and input parameters with its output. \texttt{Formaline}
was pioneered by Erik Schnetter for the Cactus Computational Toolkit
(\url{http://www.cactuscode.org}). In its original form,
\texttt{Formaline} stored a \texttt{tar.gz} file of the Cactus source
code in the Cactus executable. Upon execution, the \texttt{tar.gz}
file is written to the output directory. We have extended this concept
to include the SRO source code and all input files in \texttt{HDF5}
SRO EOS tables.

\subsection{\texttt{Formaline}-preserved Code and Inputs in \texttt{HDF5} Tables.}

\texttt{SNA} tables contain the \texttt{HDF5} datasets
\begin{itemize}
\item \texttt{SNA-skyrme.in} -- the \texttt{SNA} Skyrme input file \texttt{\color{red}input/skyrme.in}, discussed in Section~\ref{sssec:skyrme.in}.
\item \texttt{SNA-space.in} -- the \texttt{SNA} table ranges and resolution input file \texttt{\color{red}input/space.in}, discussed in Section~\ref{sssec:SNA_phase_space_in}.
\item \texttt{SNA-src.tar.gz} -- the complete \texttt{SNA} source
  tree, \texttt{make.inc}, Makefile, and git hash.
\end{itemize}

\texttt{NSE} tables contain the \texttt{HDF5} datasets
\begin{itemize}
  
\item \texttt{NSE-isotope-list.in} -- the list of isotopes used for
  the \texttt{NSE} table, file \texttt{\color{red}input/list.in}, discussed in Section~\ref{sssec:NSE_isotopes}.
\item \texttt{NSE-isotope-properties.in} -- isotope data table, file \texttt{\color{red}input/isotopes.in}, discussed in Section~\ref{sssec:mass_table.in}.
\item \texttt{NSE-output-list.in} -- the \texttt{NSE} table output namelist file, \texttt{\color{red}input/output.in}, discussed in Section~\ref{sssec:NSE_phase_space_in}     
\item \texttt{NSE-partition.in} -- -- isotope partition function data table, file \texttt{\color{red}input/partition.in}, discussed in Section~\ref{sssec:mass_table.in}.
\item \texttt{NSE-space.in} -- the \texttt{NSE} table ranges and resolution input file \texttt{\color{red}input/space.in}, discussed in Section~\ref{sssec:NSE_phase_space_in}.
\item \texttt{NSE-src.tar.gz} -- the complete \texttt{NSE} source tree, \texttt{make.inc}, Makefile, and git hash.
\end{itemize}

The \texttt{MERGE} code (see Section~\ref{sec:merge}) generates the
final EOS tables. It copies the above datasets from the input
\texttt{SNA} and/or \texttt{NSE} tables and places them in the output
\texttt{HDF5} table. It adds the following datasets specific to the \texttt{MERGE} code:
\begin{itemize}

\item \texttt{MERGE-tables.in} -- \texttt{MERGE} table output details and paths to \texttt{SNA} and \texttt{NSE} tables being merged, file \texttt{\color{red}input/tables.in}, discussed in Section~\ref{sssec:Merge_tables}. 
\item \texttt{MERGE-space.in} --  the \texttt{MERGE} table ranges and resolution input file \texttt{\color{red}input/space.in}, discussed in Section~\ref{sssec:MERGE_phase}.
\item \texttt{MERGE-transition.in} -- transition parameters used to merge the \texttt{SNA} and \texttt{NSE} tables, file \texttt{\color{red}input/transition.in}, discussed in Section~\ref{sssec:Merge_transition}. 
\item \texttt{MERGE-src.tar.gz} -- the complete \texttt{MERGE} source tree, \texttt{make.inc}, Makefile, and git hash.

\end{itemize}


\subsection{Extracting \texttt{Formaline}-preserved Information}

The SRO EOS comes with a simple Formaline extraction code written in
C++. It resides in the \texttt{\color{red}formaline\_extract} directory
in the EOS root directory and uses the EOS-wide \texttt{make.inc}.

\textbf{Compiling.}\hspace*{1em} Go into
\texttt{\color{red}formaline\_extract} and type \texttt{make}. This
creates the exectuable \texttt{\color{magenta}extract\_from\_h5}.


\textbf{Usage.}\hspace*{1em} It's as simple as it gets:

\texttt{./extract\_from\_h5 [HDF5 filename] [Dataset Name]}

Recall that you can always use \texttt{h5ls} to learn about the
contents of an \texttt{HDF5} file.



\newpage
\bibliographystyle{unsrt}
\bibliography{User_Guide}

\end{document}

